% --------------------------------------------------------------
% This is all preamble stuff that you don't have to worry about.
% Head down to where it says "Start here"
% --------------------------------------------------------------
 
\documentclass[13pt]{article}
 
\usepackage[margin=1in]{geometry} 
\usepackage{amsmath,amsthm,amssymb}
\usepackage{graphicx}
 
\newcommand{\N}{\mathbb{N}}
\newcommand{\Z}{\mathbb{Z}}
 
\newenvironment{theorem}[2][Theorem]{\begin{trivlist}
\item[\hskip \labelsep {\bfseries #1}\hskip \labelsep {\bfseries #2.}]}{\end{trivlist}}
\newenvironment{lemma}[2][Lemma]{\begin{trivlist}
\item[\hskip \labelsep {\bfseries #1}\hskip \labelsep {\bfseries #2.}]}{\end{trivlist}}
\newenvironment{exercise}[2][Exercise]{\begin{trivlist}
\item[\hskip \labelsep {\bfseries #1}\hskip \labelsep {\bfseries #2.}]}{\end{trivlist}}
\newenvironment{problem}[2][Problem]{\begin{trivlist}
\item[\hskip \labelsep {\bfseries #1}\hskip \labelsep {\bfseries #2.}]}{\end{trivlist}}
\newenvironment{question}[2][Question]{\begin{trivlist}
\item[\hskip \labelsep {\bfseries #1}\hskip \labelsep {\bfseries #2.}]}{\end{trivlist}}
\newenvironment{corollary}[2][Corollary]{\begin{trivlist}
\item[\hskip \labelsep {\bfseries #1}\hskip \labelsep {\bfseries #2.}]}{\end{trivlist}}

\newenvironment{solution}{\begin{proof}[Solution]}{\end{proof}}
 
\begin{document}
 
% --------------------------------------------------------------
%                         Start here
% --------------------------------------------------------------
 
\title{\underline{Plots of Time Vs Number of Elements}}
\author{Abhinav Kumar}

\maketitle

{\Large{\bf{The plots for each thread configuration:}}

\bigskip
Number of Threads=1:
\begin{figure}[h]
\includegraphics[scale=1]{gr1_1.eps}
\end{figure}

The above plot is when 1 thread is working on the array.

\newpage
Number of Threads=2:
\begin{figure}[h]
\includegraphics[scale=1]{gr1_2.eps}
\end{figure}

The above plot is when 2 threads are working on the array.

The threads work in parallel thereby decreasing the time.

\newpage
Number of Threads=4:
\begin{figure}[h]
\includegraphics[scale=1]{gr1_3.eps}
\end{figure}

The above plot is when 4 threads are working on the array.

The performance is considerably improved.

\newpage
Number of Threads=8:
\begin{figure}[h]
\includegraphics[scale=1]{gr1_4.eps}
\end{figure}

The above plot is when 8 threads are working on the array.

This suffers initially as thread creation time makes it slow

but gives fast result as number of elements increases.

\newpage
Number of Threads=16:
\begin{figure}[h]
\includegraphics[scale=1]{gr1_5.eps}
\end{figure}

The above plot is when 16 threads are working on the array.

Same argument as when number of threads was 8.

\newpage
Relative Comparison of Time Taken:
\begin{figure}[h]
\includegraphics[scale=1]{gr2.eps}
\end{figure}

The above lineplot is of all the threads on the same graph

for comparison of the performance. Threads 4 performs extremely

well.

\newpage
Histogram Representation of Speedup:
\begin{figure}[h]
\includegraphics[scale=1]{gr3_1.eps}
\end{figure}

Leftmost bar for thread 1 and rightmost for threads 16

\newpage
Histogram Representation of Speedup-Coloured:
\begin{figure}[h]
\includegraphics[scale=1]{gr3_2.eps}
\end{figure}

Leftmost bar for thread 1 and rightmost for threads 16

\newpage
Histogram Representation of Speedup with error bars:
\begin{figure}[h]
\includegraphics[scale=1]{gr3_3.eps}
\end{figure}

Leftmost bar for thread 1 and rightmost for threads 16.
And the variances mark the top and bottom for every bar.
 
% --------------------------------------------------------------
%     You don't have to mess with anything below this line.
% --------------------------------------------------------------
 
\end{document}

